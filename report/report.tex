%%%%%%%%%%%%%%%%%%%%%%%%%%%%%%%%%%%%%%%%%
% Short Sectioned Assignment
% LaTeX Template
% Version 1.0 (5/5/12)
%
% This template has been downloaded from:
% http://www.LaTeXTemplates.com
%
% Original author:
% Frits Wenneker (http://www.howtotex.com)
%
% License:
% CC BY-NC-SA 3.0 (http://creativecommons.org/licenses/by-nc-sa/3.0/)
%
%%%%%%%%%%%%%%%%%%%%%%%%%%%%%%%%%%%%%%%%%

%----------------------------------------------------------------------------------------
%	PACKAGES AND OTHER DOCUMENT CONFIGURATIONS
%----------------------------------------------------------------------------------------

\documentclass[paper=a4, fontsize=11pt]{scrartcl} % A4 paper and 10pt font size

\usepackage[T1]{fontenc} % Use 8-bit encoding that has 256 glyphs
%\usepackage{fourier} % Use the Adobe Utopia font for the document - comment this line to return to the LaTeX default
\usepackage[english]{babel} % English language/hyphenation
\usepackage{amsmath,amsfonts,amsthm} % Math packages
\usepackage[toc,page]{appendix}
\usepackage{lipsum} % Used for inserting dummy 'Lorem ipsum' text into the template
\usepackage{hyperref}
\usepackage{sectsty} % Allows customizing section commands
\usepackage{listings}
\usepackage{color}

\usepackage{geometry}
 \geometry{
 a4paper,
 left=15mm,
 right=15mm,
 top=15mm,
 bottom=20mm
 }

\usepackage{sectsty}
\sectionfont{\large}

 \usepackage{graphicx}
\graphicspath{ {images/} }
\usepackage{dirtytalk}
\usepackage{csquotes}

\usepackage{indentfirst}

\setlength\parindent{0pt} % Removes all indentation from paragraphs - comment this line for an assignment with lots of text
\setlength{\columnsep}{0.9cm}
%----------------------------------------------------------------------------------------
%	TITLE SECTION
%----------------------------------------------------------------------------------------
\begin{document}
\begin{titlepage}

\newcommand{\HRule}{\rule{\linewidth}{0.5mm}} % Defines a new command for the horizontal lines, change thickness here

\center % Center everything on the page

%----------------------------------------------------------------------------------------
%	HEADING SECTIONS
%----------------------------------------------------------------------------------------

\textsc{\LARGE UCD Smurfit Graduate School of Business}\\[1.5cm] % Name of your university/college
\includegraphics[scale = 0.6]{images/logo.png} \\ [1cm]
\textsc{\Large Network Software Modelling}\\[0.5cm] % Major heading such as course name

%----------------------------------------------------------------------------------------
%	TITLE SECTION
%----------------------------------------------------------------------------------------

\HRule \\[0.4cm]
{ \LARGE \bfseries Assignment One - Dijkstra algorithm}\\[0.4cm] % Title of your document
\HRule \\[1.5cm]

%----------------------------------------------------------------------------------------
%	AUTHOR SECTION
%----------------------------------------------------------------------------------------

\begin{minipage}{0.4\textwidth}
\begin{flushleft} \large
\emph{Authors:}\\
\small{Conor \textsc{Reid}\\ (16202630 - MSc B.A Part-time)} % Your name
\end{flushleft}
\end{minipage}
~
\begin{minipage}{0.4\textwidth}
\begin{flushright} \large
\emph{Lecturer:} \\
Dr. James \textsc{McDermott} % Supervisor's Name
\end{flushright}
\end{minipage}\\[4cm]

%----------------------------------------------------------------------------------------
%	DATE SECTION
%----------------------------------------------------------------------------------------

\vspace{5cm}

{\large \today}\\[3cm] % Date, change the \today to a set date if you want to be precise

\vfill % Fill the rest of the page with whitespace

\end{titlepage}
\clearpage
\setcounter{tocdepth}{1}
\tableofcontents
\clearpage
%----------------------------------------------------------------------------------------
%	MAIN BODY
%----------------------------------------------------------------------------------------
%--------------------------------
%	Overview
%--------------------------------
\twocolumn
\section{Dijkstra's Algorithm}
{\bf Briefly describe Dijkstra's algorithm for solving the shortest path problem
(which we have covered in class), including a statement of its time complexity.}\\\\
{The answer}\\\\
{Dijkstra's Algorithm is among the most fundamental algorithms in modern computer science, used
for finding the path of lowest cost between a given point in the graph, and all others. More
formally; Given a vertex $s$ in a weighted directed graph $G = (V,E)$ where edges are non-negitive, Dijkstra's Algorithm
finds the path of lowest cost from $s$ to all other vertices in G. This is sometimes referred to as the
{\it single-source shortest path problem}. \\\\
The algorithm does this as follows; all nodes are assigned some cost value, set to zero for the initial node
and infinity for the rest. All vertices, except the initial one, are passed into a set which is
cycled through. The neighbors of the initial node are considered first, taking the cost (or distance)
associated with that journey between the nodes.}


\section{Bidirectional Dijkstra's Algorithm}
{\bf Describe the bidirectional Dijkstra algorithm – a variant which is more efficient in practice}
{\it In order to understand it, you can research it in textbooks, MOOCs, blogs,
or other external resources. State any assumptions needed for the algorithm to work.}\\\\
{The answer}

\section{Dijkstra's Original Algorithm - Comparison}
{\bf With the aid of a diagram, explain how it differs from Dijkstra's original algorithm.}\\\\
{The answer}


\section{Time Complexity}
{\bf State its time complexity and explain why it is more efficient.}\\\\
{The answer}

\section{Implementation}
{\bf Implement both Dijkstra's algorithm and the bidirectional variant, using appropriate data structures for efficiency.}\\\\
{The answer}

\section{Run-Time behavior}
{\bf Test the run-time behavior of both algorithms on randomly generated graphs of varying
sizes in order to demonstrate their scaling behavior. Include a table of data showing run-times.}\\\\
{The answer}

\section{Conclusions}
{\bf Briefly state your conclusion concerning run-time behavior.}\\\\
{The answer}

\clearpage
\begin{thebibliography}{9}
\end{thebibliography}
\end{document}
