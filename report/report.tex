%%%%%%%%%%%%%%%%%%%%%%%%%%%%%%%%%%%%%%%%%
% Short Sectioned Assignment
% LaTeX Template
% Version 1.0 (5/5/12)
%
% This template has been downloaded from:
% http://www.LaTeXTemplates.com
%
% Original author:
% Frits Wenneker (http://www.howtotex.com)
%
% License:
% CC BY-NC-SA 3.0 (http://creativecommons.org/licenses/by-nc-sa/3.0/)
%
%%%%%%%%%%%%%%%%%%%%%%%%%%%%%%%%%%%%%%%%%

%----------------------------------------------------------------------------------------
%	PACKAGES AND OTHER DOCUMENT CONFIGURATIONS
%----------------------------------------------------------------------------------------

\documentclass[paper=a4, fontsize=11pt]{scrartcl} % A4 paper and 10pt font size

\usepackage[T1]{fontenc} % Use 8-bit encoding that has 256 glyphs
%\usepackage{fourier} % Use the Adobe Utopia font for the document - comment this line to return to the LaTeX default
\usepackage[english]{babel} % English language/hyphenation
\usepackage{amsmath,amsfonts,amsthm} % Math packages
\usepackage[toc,page]{appendix}
\usepackage{lipsum} % Used for inserting dummy 'Lorem ipsum' text into the template
\usepackage{hyperref}
\usepackage{sectsty} % Allows customizing section commands
\usepackage{listings}
\usepackage{color}

\usepackage{geometry}
 \geometry{
 a4paper,
 left=15mm,
 right=15mm,
 top=15mm,
 bottom=20mm
 }
 \usepackage{indentfirst}


\usepackage{sectsty}
\sectionfont{\large}

 \usepackage{graphicx}
\graphicspath{ {images/} }
\usepackage{dirtytalk}
\usepackage{csquotes}

\setlength\parindent{0pt} % Removes all indentation from paragraphs - comment this line for an assignment with lots of text
\setlength{\columnsep}{0.9cm}
%----------------------------------------------------------------------------------------
%	TITLE SECTION
%----------------------------------------------------------------------------------------
\begin{document}
\begin{titlepage}

\newcommand{\HRule}{\rule{\linewidth}{0.5mm}} % Defines a new command for the horizontal lines, change thickness here

\center % Center everything on the page

%----------------------------------------------------------------------------------------
%	HEADING SECTIONS
%----------------------------------------------------------------------------------------

\textsc{\LARGE UCD Smurfit Graduate School of Business}\\[1.5cm] % Name of your university/college
\includegraphics[scale = 0.6]{images/logo.png} \\ [1cm]
\textsc{\Large Network Software Modelling}\\[0.5cm] % Major heading such as course name

%----------------------------------------------------------------------------------------
%	TITLE SECTION
%----------------------------------------------------------------------------------------

\HRule \\[0.4cm]
{ \LARGE \bfseries Assignment One - Dijkstra's Algorithm}\\[0.4cm] % Title of your document
\HRule \\[1.5cm]

%----------------------------------------------------------------------------------------
%	AUTHOR SECTION
%----------------------------------------------------------------------------------------

\begin{minipage}{0.4\textwidth}
\begin{flushleft} \large
\emph{Author:}\\
\small{Conor \textsc{Reid}\\ (16202630 - MSc B.A Part-time)} % Your name
\end{flushleft}
\end{minipage}
~
\begin{minipage}{0.4\textwidth}
\begin{flushright} \large
\emph{Lecturer:} \\
Dr. James \textsc{McDermott} % Supervisor's Name
\end{flushright}
\end{minipage}\\[4cm]

%----------------------------------------------------------------------------------------
%	DATE SECTION
%----------------------------------------------------------------------------------------

\vspace{5cm}

{\large \today}\\[3cm] % Date, change the \today to a set date if you want to be precise

\vfill % Fill the rest of the page with whitespace

\end{titlepage}
\clearpage
%----------------------------------------------------------------------------------------
%	MAIN BODY
%----------------------------------------------------------------------------------------
%--------------------------------
%	Overview
%--------------------------------
\twocolumn
\setlength{\parindent}{10ex}
{\bf \noindent Dijkstra's Algorithm}
{\par \noindent  Dijkstra's Algorithm is used for finding the shortest path (or that of lowest cost) between a initial point in a directed graph, and all others. More formally; Given an initial vertex $s$ in a weighted directed graph $G = (V,E)$ where edges are non-negative, Dijkstra's Algorithm finds the path of lowest cost from $s$ to all other vertices in G. This is sometimes referred to as the {\it single-source shortest path problem}. 
\par The algorithm works as follows; Every node in a graph is assigned a tentative distance value, which zero for the initial node and infinity for all others. The initial node is marked the current node and thus visited, all others are marked as unvisited and are passed into a set. Visited nodes are not visited again. The neighbours of the current node are calculated for actual tentative distance. That is, the distance to the second node going through the first is calculated. If this distance is lower than the previously assigned tentative distance to this node then it overwrites the same. The process is repeated, assigning the next unvisited node with the lowest tentative distance as the current node. The algorithm stops when the unvisited set is empty.
\par Dijkstra's algorithm has time complexity $O(n^2)$ where $n$ is the number of nodes in the graph. }\\\\
{\bf Bidirectional Dijkstra's Algorithm}\\
{The bi-directional version of Dijkstra's algorithm can be used in an effort to "speed up" the searching process, by reducing the number of visited vertices. In the above explanation, the searching process happens in the direction of $s$ (the initial node) to $t$ (the target node). In this version, searching direction alternates between forward and backwards. That is, we alternate between searching from $s$ to $t$ and from $t$ to $s$, the latter involving following edges backwards. We denote distances for forward search as $d_f(u)$ and distances for backward search as $d_b(u)$. The algorithm terminates when some vertex has been removed from the queues of both searches. Intuitively, we can view bi-directional search as the rowing of two bubbles around the initial vertex and target vertex, terminating when these bubbles intersect \cite{mit}.} 
\par Interesting to note, is that this technique does not improve the worst case behaviour of the algorithm, or its big 0 notation. That is, bi-drectional searching still has time complexity of $O(n^2)$ (which is really $O(\frac{n^2}{2})$ which is $\sim$ $O(n^2)$). \\\\
{\bf Dijkstra's Original Algorithm - Comparison}\\
{\bf Time Complexity}\\
{\bf Implementation}\\
{\bf Run-Time behavior}\\
{\bf Conclusions}
\clearpage
\onecolumn
\begin{thebibliography}{9}
\bibitem{mit} \url{https://courses.csail.mit.edu/6.006/fall11/lectures/lecture18.pdf}
\end{thebibliography}
\end{document}
